% Waiting for Eloïse report template
% We need better display
% Sommaire inherited from projet_SGBD_rendu19.pdf

\documentclass[a4paper,10.5pt]{article}
\usepackage[utf8]{inputenc}
\usepackage[T1]{fontenc}
\usepackage[french]{babel}
\usepackage[top=2cm, bottom=2cm, left=2cm, right=2cm]{geometry}
\usepackage{hyperref}

%pictures
\usepackage{graphicx}

%maths
\usepackage{amsmath}
\usepackage{amssymb}
\usepackage{mathtools}
\usepackage{empheq}



\begin{document}
\noindent
\begin{minipage}{0.20\textwidth}
\includegraphics[width=\textwidth]{logo}
\end{minipage}
\hfill
\begin{minipage}{0.71\textwidth}
Système de gestion de Base de Donnée \hfill Nov. 2019\\

\begin{center}
{\Large \textbf{Projet}}

\vspace{0.5em}
 \large Melvin \bsc{Even} \quad Pierre \bsc{Pavia} \quad Maëlle \bsc{Toy-Riont-Le-Dosseur} \quad Lucas \bsc{Henry}
\end{center}\vspace{0.3em}


\end{minipage}\\

\noindent
\rule{\linewidth}{0.5mm}
\setcounter{section}{1}

\section{Modélisation des données}
\subsection{Description du contexte de l'application}
\subsection{Modèle entité-association}
\subsection{Liste des opérations prévues sur la base}

Voici la liste des actions retenues sur la base de donnée~:

Consultation~:

\begin{itemize}
  \item Liste des cartes d’un certain type.
  \item Liste des cartes qui ne font partie d’aucun deck.
  \item La liste des joueurs qui n’ont fait aucune partie (ce sont juste des collectionneurs).
\end{itemize}

Statistiques~:

\begin{itemize}
  \item La liste des joueurs, avec le nombre de cartes qu’ils possèdent.
  \item La liste des joueurs classés par ordre décroissant selon la valeur de
    leur collection (la valeurd’une carte étant estimée au produit de sa côte
    par son état).
  \item La liste des cartes avec le nombre de joueurs qui les utilisent dans
    leurs decks.
  \item La liste des joueurs possédant le plus de cartes rares (date
    d’impression antérieure à 2000 ou tirage inférieur à 100).
  \item La liste des familles de carte, avec, pour chaque famille, la
    caractéristique dans laquelle cette famille a le meilleur niveau.
\end{itemize}

\section{Schéma relationnel}
\subsection{Passage au relationnel}

La conversion des entités donne rapidement les premières règles suivantes :\\
\\
\texttt{Cartes(\underline{id\_carte}, titre, description, famille, attaque, defense)} \\
\texttt{Versions(\underline{id\_version}, \#id\_carte, rendu, cote, tirage)} \\
\texttt{Joueurs(\underline{pseudo}, nom\_joueur, prenom\_joueur)} \\
\texttt{Possessions(\underline{id\_possession}, \#pseudo, \#id\_version, date\_acquisition, mode\_achat,\\ date\_vente, prix\_vente, etat)} \\
\texttt{Decks(\underline{id\_deck}, nom\_deck, \#pseudo)} \\
\texttt{Parties(\underline{id\_partie}, date\_partie, lieu\_partie, type\_tournoi, resultat\_partie)} \\

Ensuite, en observant les cardinalités, on se rend compte qu'il est nécessaire de créer les deux relations supplémentaires suivantes. \\
\\
\texttt{Jeu(\underline{id\_partie}, \underline{id\_deck}, \underline{pseudo})} \\
\texttt{Appartenance(\underline{id\_possession}, \underline{id\_deck})} \\

\subsection{Contraintes d'intégrité, dépendances fonctionnelles}
Après l'ajout des contraintes : \\
\\
\texttt{Cartes(\underline{id\_carte} [U, NN], titre, description, famille, attaque, defense)} \\
\texttt{Versions(\underline{id\_version} [U, NN], \#id\_carte [U, NN], rendu, cote, tirage)} \\
\texttt{Joueurs(\underline{pseudo} [U, NN], nom\_joueur, prenom\_joueur)} \\
\texttt{Possessions(\underline{id\_possession} [U, NN], \#pseudo [U, NN], date\_acquisition, mode\_achat, date\_vente, prix\_vente, etat)} \\
\texttt{Decks(\underline{id\_deck} [U, NN], nom\_deck, \#pseudo [U, NN])} \\
\texttt{Appartenance(\underline{id\_possession} [U, NN], \underline{id\_deck} [U, NN])} \\
\texttt{Parties(\underline{id\_partie} [U, NN], date\_partie, lieu\_partie, type\_tournoi, resultat\_partie)} \\
\texttt{Jeu(\underline{id\_partie} [U, NN], \underline{id\_deck} [U, NN], \underline{pseudo} [U, NN])} \\

\textit{Contraintes : U = Unique, NN = Non nul}
% contrainte pour la table Appartenance : les entités représentées par les clés étrangères #id_possession #id_deck doivent contenir la même clé étrangère #pseudo 

\subsection{Shéma relationnel en $3^e$ forme normale}

\section{Implantation}
\subsection{Création de la base de donnée}
\subsection{Implémentation des commandes SQL}

\section{Utilisation}
\subsection{Description de l'environnement d'exécution}
\subsection{Notice d'utilisation}
\subsection{Description des interfaces éventuelles}

\end{document}
