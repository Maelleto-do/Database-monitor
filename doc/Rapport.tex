% Waiting for Eloïse report template
% We need better display
% Sommaire inherited from projet_SGBD_rendu19.pdf

\documentclass{article}
\usepackage[french]{babel}
\usepackage[utf8]{inputenc}

\begin{document}

\section{Modélisation des données}
\subsection{Description du contexte de l'application}
\subsection{Modèle entité-association}
\subsection{Liste des opérations prévues sur la base}

\section{Schéma relationnel}
\subsection{Passage au relationnel}

La conversion des entités donne rapidement les premières règles suivantes :\\
\\
\texttt{Cartes(\underline{id\_carte}, titre, description, famille, attaque, defense)} \\
\texttt{Versions(\underline{id\_version}, \#id\_carte, rendu, cote, tirage)} \\
\texttt{Joueurs(\underline{pseudo}, nom\_joueur, prenom\_joueur)} \\
\texttt{Possessions(\underline{id\_possession}, \#pseudo, \#id\_version, date\_acquisition, mode\_achat, date\_vente, prix\_vente, etat)} \\
\texttt{Decks(\underline{id\_deck}, nom\_deck)} \\
\texttt{Parties(\underline{id\_partie}, date\_partie, lieu\_partie, type\_tournoi, resultat\_partie)} \\

Ensuite, en observant les cardinalités, on se rend compte qu'il est nécessaire de créer les deux relations supplémentaires suivantes. \\
\\
\texttt{Jeu(\underline{id\_partie}, \underline{id\_deck}, \underline{pseudo})} \\
\texttt{Appartenance(\underline{id\_possession}, \underline{id\_deck})} \\

\subsection{Contraintes d'intégrité, dépendances fonctionnelles}
\subsection{Shéma relationnel en $3^e$ forme normale}

\section{Implantation}
\subsection{Création de la base de donnée}
\subsection{Implémentation des commandes SQL}

\section{Utilisation}
\subsection{Description de l'environnement d'exécution}
\subsection{Notice d'utilisation}
\subsection{Description des interfaces éventuelles}

\end{document}
